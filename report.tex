\documentclass[12pt]{article}
\usepackage[margin=1.25in]{geometry}

\begin{document}
\title{Applying Spin and Angular Resolved Photoemission Spectroscopy to the study of Spin Orbit Coupling in Materials
\\ \Large{\emph{Comps II Proposal}}}
\author{Gregory Petropoulos}
\date{September 12, 2012}
\maketitle

\section{Introduction}
Angular resolved photoemission spectroscopy (ARPES) is an experimental method that allows physicists to directly probe the surface band structure of materials.
This is accomplished by shining light on a sample and measuring the energy and momentum of photoemitted electrons \cite{Damascelli}.
ARPES has been successful in studying a wide range of condensed matter systems including:  direct band mapping of solids,  high temperature superconductors, and topological insulators.
More recent developments in the field of ARPES have allowed researchers to measure the spin of photoemitted electrons \cite{Dil, Osterwalder} opening up the field of spin resolved ARPES.
Spin measurements can be difficult experimentally and spin resolved ARPES is complementary to other techniques and is especially useful in studying itinerant magnetic systems \cite{Osterwalder}.
Understanding spin dynamics in condensed matter systems is especially important for the physics and application of devices that exploit electric spin, called spintronics \cite{wolf}.

Spin resolved ARPES is currently being used to measure various strong spin-orbit materials by mapping their ground state spin-orbit texture.  Spin-orbit coupling is the interaction of a particle's spin with its momentum; this coupling in an asymmetrical confining potential can lead to an asymmetry in the spin polarization of the material.  The Rashba effect causes spin asymmetry to arise in a material due to differences in the structure between the bulk and surface of a material \cite{Dil}.  In both types of materials, the spin of photoemitted electrons contains information about the spin asymmetry in the sample.  The asymmetry of the photoemitted electrons is then measured using Mott scattering, which separates spin states by scattering the photoemitted electron beam off the Coulomb field of heavy atoms.

I propose an article describing the physics of spin resolved ARPES and its application to studying Spin Orbit Coupling in Materials.  My review will discuss the experimental set up of spin resolved ARPES and \lowercase{spin orbit coupling in materials}.  My area of research is theoretical particle physics, specifically using lattice gauge theory to study beyond standard model physics.


\section{Spin Orbit Couplings in Materials}
The electronic spin degrees of freedom were first discovered in the Stern Ger... experiment.
In this experiment (explain experment and results from Shankar)
\subsection{Rashba}
\section{Angular resolved photoemission spectroscopy}
\subsection{Looking for Spin Effects}
\subsection{Experimental Setups}
\section{Applications to Spintronics}
One of the most exciting applications of studying the surface spin properties of materials is the application of these materials for spintronics.
Spintronics are electronic devices that use the



\section{}

\bibliographystyle{plain}
\bibliography{mybib}

\end{document}
