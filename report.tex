\documentclass[12pt]{article}
\usepackage[margin=1.25in]{geometry}
\usepackage{amsmath}
\usepackage{amssymb}
\usepackage{graphicx}
\usepackage{gensymb}
\usepackage{bbold}
\usepackage{sectsty}
\sectionfont{\large}
\usepackage{setspace}
\doublespacing
\usepackage[sorting=none]{biblatex}
\bibliography{mybib}

\newcommand{\newln}{\\&{}}
\newcommand{\parenthnewln}{\right.\\&\left.\quad\quad{}}

\begin{document}
\title{Applying Spin and Angular Resolved Photoemission Spectroscopy to the Study of the Rashba Effect in Bismuth Films}
\author{Gregory Petropoulos}
\maketitle
\abstract{
This review opens with a discussion of spin-orbit coupling and how the Rashba effect can lead to spin-split band structure.
The focus of this paper is the experimental setup of spin and angular resolved photoemission spectroscopy.
I end with a discussion of recent results for bismuth films and an outlook for the future.
}

\section{Introduction}
Angular resolved photoemission spectroscopy (ARPES) is an experimental method that allows physicists to directly probe the surface band structure of materials.
This is accomplished by shining light on a sample and measuring the energy and momentum of photo-emitted electrons \cite{Damascelli}.
ARPES has been successful in studying a wide range of condensed matter systems including:  direct band mapping of solids,  high temperature superconductors, and topological insulators.
More recent developments in the field of ARPES have allowed researchers to measure the spin of photo-emitted electrons \cite{Dil, Osterwalder} opening up the field of spin resolved ARPES.
Spin measurements can be difficult experimentally and spin resolved ARPES is complementary to other techniques and is especially useful in studying itinerant magnetic systems \cite{Osterwalder}.

Spin resolved ARPES (SARPES) is currently being used to measure various strong spin-orbit materials by mapping their ground state spin-orbit texture.
Spin-orbit coupling is the interaction of a particle's spin with its momentum; this coupling in an asymmetrical confining potential can lead to an asymmetry in the spin polarization of the material.
The Rashba effect causes spin asymmetry to arise in a material due to differences in the structure between the bulk and surface of a material \cite{Dil}.
Understanding spin dynamics in condensed matter systems is especially important for the physics and application of devices that exploit electric spin, called spintronics \cite{wolf}.

\section{Spin Orbit Coupling}
Spin orbit coupling occurs when an electron's spin interacts with its momentum.
This effect happens in atomic systems and is responsible for the fine structure in atomic spectra.
While there are semi-classical derivations of spin-orbit coupling, the process is best described by a non-relativistic expansion of the Dirac equation \cite{Dirac, Winkler}
\begin{align}\begin{split}
  \label{eq:Dirac}
  & (c\boldsymbol{\alpha\cdot p}+\beta m_{0}c^2+V)\psi=E\psi \quad\text{with}\newln\boldsymbol{\alpha}=\left(\begin{array}{cc} 0 & \boldsymbol{\sigma} \\ \boldsymbol{\sigma} & 0 \end{array} \right) \quad \beta=\left(\begin{array}{cc} \mathbb{1} & 0 \\ 0 & -\mathbb{1} \end{array} \right)\text{.}
\end{split}\end{align}
In Eq. (\ref{eq:Dirac}) $\boldsymbol\sigma$ are the Pauli matrices, $\mathbb{1}$ is the two-by-two identity matrix, and $\psi$ is a four-component Dirac spinor.
For non-relativistic electrons, $\frac{v}{c}\lesssim 0.1$, an expansion in $(\frac{v}{c})^2$ is appropriate.
To first order we get
\begin{align}\begin{split}
  \label{eq:expansion}
& \left [ \frac{p^2}{2m_0}\right.+V+\frac{e\hbar}{2m_0}\boldsymbol{\sigma\cdot B}-\frac{e\hbar}{4m_0^2c^2}\boldsymbol{\sigma\cdot p \times E}-\frac{e\hbar^2}{8m_0^2c^2}\boldsymbol{\nabla\cdot E} \newln-\frac{p^4}{8m_0^3c^2}-\frac{e\hbar p^2}{4m_0^3c^2}\boldsymbol{\sigma \cdot B}-\left.\frac{(ehB)^2}{8m_0^3c^2}\right]\psi_n=E\psi_n\text{,}
\end{split}\end{align}
where $\psi_n$ has been normalized to preserve unitarity in the expansion, $\boldsymbol{B}=\boldsymbol{\nabla}\times\boldsymbol{A}$, and $\boldsymbol{E}=\frac{1}{e}\nabla V$.
The expansion in Eq. (\ref{eq:expansion}) contains several well known terms from non-relativistic quantum mechanics.
The third term is the Zeeman term, the fourth term is the spin-orbit coupling, and the fifth is the Darwin term.
The last three terms are first order corrections.
The spin-orbit term can be rewritten in a more familiar way since $\boldsymbol{E}=\frac{e\hat{r}}{r^2}$ is the electric field sourced by the nucleus.
\begin{align}
  \label{eq:soc}
  -\frac{e\hbar\boldsymbol{\sigma\cdot p \times E}}{4m_0^2c^2}=-\frac{e^2\hbar\boldsymbol{\sigma\cdot p \times r}}{4m_0^2c^2r^3}\newline=\frac{\boldsymbol{\hat{S}\cdot \hat{L}}}{2m_0^2c^2r^3}
\end{align}
Here $\hat{\boldsymbol{S}}, \hat{\boldsymbol{L}}$ are the spin operator and angular momentum operator respectively and the Thomas factor of $\frac{1}{2}$ appears naturally.

The above derivation shows that spin-orbit coupling becomes larger as the electron's momentum increases.
The effect is also enhanced by strong $\boldsymbol{E}$ fields near high Z nuclei.
Since both of these conditions are met in crystalline solids it is not surprising to find spin-orbit coupling in materials.
In crystalline solids the periodic arrangement of the nuclei give rise to an electronic band structure.
These bands are then dressed by more complicated phenomena such as spin-orbit couplings, surface effects, and multi body physics.
Since the band structure of a solid contains information about its physical properties, especially the band structure near the Fermi surface, it is important to understand their structure in detail.

\section{The Rashba Effect}
The Bychkov-Rashba model describes the motion of an electron in a two-dimensional electron gas with a potential gradient normal to the plane of the gas \cite{Rashba}.
In a crystal, time-reversal and space-inversion symmetry require the electron energy eigenvalue states to obey $E(\boldsymbol{k},\uparrow)=E(-\boldsymbol{k},\downarrow)$ and $E(\boldsymbol{k},\downarrow)=E(-\boldsymbol{k},\downarrow)$ respectively \cite{Okuda-Kimura}.
In the absence of magnetic fields this leads to Kramer's degeneracy
\begin{align}
  \label{eq:kramer}
  E(\boldsymbol{k},\uparrow)=E(\boldsymbol{k},\downarrow)\text{.}
\end{align}
The truncation of the periodic crystal in the normal direction at the surface breaks spatial inversion symmetry.
If spatial inversion symmetry breaks then Eq. (\ref{eq:kramer}) breaks with it; resulting in a spin-splitting of electronic band structures \cite{Winkler}.

This splitting can also be thought of in a more concrete way that elucidates some of its characteristics.
If we give our two dimensional system a crystal potential $V$, the lack of inversion symmetry in the normal direction to the crystal surface can generate a net electric field $\boldsymbol{E}=-\boldsymbol{\nabla}V=E_0\hat{z}$.
As shown in term four of Eq. (\ref{eq:expansion}) this $\boldsymbol{E}$ field gives rise to a nonzero spin-orbit coupling.
This magnetic field will couple to the electron's spin creating an effective spin-orbit coupling; in two dimensions this is reduced to the Rashba Hamiltonian
\begin{align}
  \label{eq:rashba}
  H_{\text{R}}=\alpha_{\text{R}}(\boldsymbol{k}\times\hat{z})\cdot\hat{\boldsymbol{S}}\text{.}
\end{align}
The factor $\alpha_{\text{R}}$ is treated as a coupling; it is well defined in this toy derivation but in real materials it is nontrivial to calculate.
Now we have the total Hamiltonian
\begin{align}
  \label{eq:totalH}
  H=H_{\text{free}}+H_{\text{R}}=\frac{\hbar^2k_\parallel^2}{2m}+\alpha_{\text{R}}(k_x\hat{S}_y-k_y\hat{S}_x)\text{,}
\end{align}
with eigenvalues
\begin{align}
  \label{eq:eigenvalues}
  E_k=\frac{\hbar^2k_{\parallel}^2}{2m}\pm\alpha_{R}\left|\boldsymbol{k}_{\parallel}\right|\text{.}
\end{align}
The wavefunctions for the two dimensional electron gas are
\begin{align}
  \label{eq:wavefn}
  \psi_{\boldsymbol{k}}=\frac{e^{i\boldsymbol{k}\cdot\boldsymbol{r}}}{2\sqrt{S}}\left(\begin{array}{c} 1 \\ \mp ie^{i\phi} \\ 0 \end{array} \right)\text{,}
\end{align}
where $\boldsymbol{k}=k(\cos\phi,\sin\phi,0)$ and $S$ is the surface area.
Now we can find the spin expectation value
\begin{align}
  \label{eq:expectation}
  \left<\psi_{\boldsymbol{k}}\left|\hat{\boldsymbol{S}}\right|\psi_{\boldsymbol{k}}\right>=\frac{1}{k}\left(\begin{array}{c} \mp k_x \\ \mp k_y \\ 0 \end{array} \right) = \left(\begin{array}{c} \mp \sin\phi \\ \mp \cos\phi \\ 0 \end{array} \right)\text{.}
\end{align}
\begin{figure}[th]
  \centering
  \includegraphics[height=2in]{fig/rashba.pdf}
  \caption[]
  {The band structure of a two-dimensional electron gas due to the Rashba effect.  The $\bar{\Gamma}$ point is an arbitrary shift to the center of symmetry in momentum space.  At some energy the split bands cross at the $\bar{\Gamma}$ point \cite{Dil}.}
  \label{fig:rashba}
\end{figure}
This result shows that the spin expectation value is always oriented perpendicular to both the electron momenta and the surface normal.
Putting this all together we get a picture shown in Fig. \ref{fig:rashba} of two spin split bands offset from a common central symmetry point (the $\bar{\Gamma}$ point).
If a cut is made to these bands at a given energy the result is a constant energy surface of two concentric circles around the $\bar{\Gamma}$ point.
The two circles have their spin polarization vector oriented along the circumference of the circle with opposite orientation.

\section{Angular Resolved Photoemission Spectroscopy}
The photoelectric effect, by which light incident on a metal liberates electrons from the material, was first discovered by Hertz in 1887.
Physicists of the day did not understand the photoelectric effect until Albert Einstein published his 1905 paper, for which he later won the Nobel Prize.
Einstein realized that the incident light came in quantum packets called photons.
These photons could be absorbed by an electron in the material, a process that transfers the photon energy and momentum to the electron.
If the electron has enough energy after absorbing a photon, it can overcome the potential barrier, which keeps electrons in the material.
This process is summarized in Eq. (\ref{eq:photoelectric}).
\begin{align}
  \label{eq:photoelectric}
  E_{\text{kin}}=h\nu-\phi
\end{align}
Where $E_{\text{kin}}$ is the maximum kinetic energy of the escaping electron, $\nu$ is the photon frequency, and $\phi$ is the work function of the material.
Work functions of typical metals are approximately 4-5 eV.

The photoemitted electrons escape from the surface in all directions.
By measuring the energy, $E_{\text{kin}}$, of photoelectrons coming off a surface at a given angle, the electron momentum $\boldsymbol{p}$ can be completely determined.
The magnitude of the photoelectron momentum is given by $|\boldsymbol{p}|=\sqrt{2mE_{\text{kin}}}$ and the parallel and perpendicular components can be obtained from the polar and azimuthal emission angle.
%include equations for parallel and perpendicular momentum
With the photon energies used in ARPES experiments the photon momentum can be neglected.
Therefore in the noninteracting electron picture with conservation of energy and momentum the kinetic energy and momentum of the photoelectron can be related to the binding energy $E_b$ and the crystal momentum $\hbar\boldsymbol{k}$ in the solid\cite{Damascelli}:
\begin{align}
  E_{\text{kin}}=h\nu-\phi-|E_b|
\end{align}
\begin{align}
  \boldsymbol{p_{\parallel}}=\hbar\boldsymbol{k_{\parallel}}=\sqrt{2mE_{\text{kin}}}\sin\theta
\end{align}
$\theta$ is the angle coming off the surface normal.
The perpendicular component of the wave vector $\boldsymbol{k_{\perp}}$ is not conserved across the sample surface because translational symmetry across the surface in the normal direction is broken.  
As a result, the determination of the full crystal wave vector \emph{k} can not be determined from ARPES alone.
In low dimensional systems $\boldsymbol{k_{\perp}}$ becomes less relevant; in the case of 2D films the electronic dispersion is almost completely characterized by $\boldsymbol{k_{\parallel}}$.

\begin{figure}[h]
  \centering
  \includegraphics[height=2in]{fig/arpes.pdf}
  \caption[]
  {Photo-emitted electrons are focused by the lens and separated in energy by the hemisphere analyser \cite{Damascelli}.}
  \label{fig:arpes}
\end{figure}
ARPES experiments need a light source that is energetic enough to cause the sample to photo-emit.
Three types of photon sources are typically used:  gas discharge lamps \cite{Damascelli}, synchrotron radiation \cite{Damascelli}, and more recently lasers \cite{Dessau}.
To date SARPES experiments have only used gas discharge lamps and synchrotron radiation.
Synchrotron radiation is advantageous because it has a wide spectral range (visible light to x-ray), the intensity is very high, and it is very polarized.
Gas discharge lamps provide high resolution because they produce relatively low energy light at a well defined frequency.

Regardless of the source, once the light strikes the sample the photo-emitted electrons are collected by the hemispherical analyzer shown in Fig. \ref{fig:arpes}.
The analyzer determines the emitted electrons emission angle and energy.
It is composed of a multi-element electrostatic input lens, a hemispherical deflector, and an electron detector.
The lens decelerates and focuses the photoelectrons onto the entrance slit to the deflector.
The deflector is made of two concentric hemispheres of radius $R_1$ and $R_2$.
The hemispheres are held at a potential difference $\Delta V$, and effectively selects out electrons within a narrow energy range centered at $E_{\text{pass}}=e\Delta V / (\frac{R_1}{R_2}-\frac{R_2}{R_1})$.
The entire ARPES apparatus is under a vacuum, typically lower than $5\times10^{-11}$ torr \cite{Damascelli}.

\section{Looking for Spin Effects}
\begin{figure}[h]
  \centering
  \includegraphics[height=2.5in]{fig/asymmetry.pdf}
  \caption[]
  {(a) In spin dependant scattering a target scatters an incident beam of intensity $I_0$ into two beams $I_A$ and $I_B$.  (b) An example of how an energy distribution curve (EDC) may look if measured from $I_0$ or $I_A$ and $I_B$.  (c) The measured spin asymmetry is much smaller than the actual spin polarization.  (d) Comparing spin integrated EDC with the spin resolved EDC \cite{Okuda-Kimura}.}
  \label{fig:asymmetry}
\end{figure}
Traditional ARPES measures both the energy and momentum degrees of freedom but ignores the spin degree of freedom.
SARPES measures all three quantum numbers simultaneously and allows for a much richer understanding of the band structure of materials.
Accomplishing this is more technically challenging than it appears at first glance.
A Stern-Gerlach experiment would be the most direct way to investigate spin polarization and can determine spin polarization with 100\% efficiency.
However, the Lorentz force makes it impossible to measure the spin of a charged object such as an electron.
As a result physicists measure electron spin indirectly by spin-dependant scattering off of a target.
As one can guess, this is a very inefficient process.
Fig. \ref{fig:asymmetry} (a) and (b) show how this process is realized.
The spin dependant scattering is characterized by an asymmetry $A$ between two scattering channels A and B \cite{Okuda-Kimura}.
\begin{align}
  A=\frac{I_A-I_B}{I_A+I_B}
\end{align}
Because the Stern-Gerlach experiment is 100\% efficient, this asymmetry is the spin polarization.
In the scattering used in experiments the efficiency is much smaller.
Instead the polarization is written as an energy dependent function such that the polarization is given by $P=S(E)A$.
The function $S(E)$ is called the Sherman function and it measures the efficiency of the target for a given electron energy.
If the Sherman function is known and the asymmetry is measured, the spin up and spin down intensities are
\begin{align}
  I_{\uparrow}=\frac{(1+P)I}{2}\quad I_{\downarrow}=\frac{(1-P)I}{2}\quad \text{with}\quad I=I_{\uparrow}+I_{\downarrow}\propto I_A+I_B\text{.}
\end{align}
An overall figure of merit can be assigned to the measurement of spin polarization $\epsilon=\frac{I}{I_0}(S(E))^2$.
The figure of merit for a SARPES set up of $1\times10^{-2}$ means that you need to collect 100 times more electrons to get the same resolution as an identical spin integrated experiment.

The most common type of spin detector uses Mott scattering.
Mott scattering takes advantage of spin-orbit coupling between relativistic electrons being scattered with atomic nuclei; shown in Fig. \ref{fig:detectortypes} (a).
Electrons are typically accelerated to relativistic energies and strike a thin gold film.
Gold is often chosen because it is easy to make, chemically inert, and has a relatively high Z.
Two beams, A and B, are then measured coming off the gold target at equal angles \cite{Hoesch}.
Mott detectors are easy to work with because their scattering properties are extremely stable.
Additionally, calibration of Mott detectors can be achieved by measuring the Sherman function and comparing to theoretical calculations.
One drawback of Mott scattering is that the two channels are measured using different detectors, leading to an unwanted asymmetry from instrumentation \cite{Okuda-Kimura}.
In magnetic samples this can be accounted for by reversing the magnetic field on the sample.
Typical figure of merit values for Mott scattering is $1\times10^{-4}$.

\begin{figure}[h]
  \centering
  \includegraphics[height=2.5in]{fig/detectortypes.pdf}
  \caption[]
  {(a) \textit{Left:}  Mott scattering off a target, scattered beams come off the target at equal angles.  \textit{Right:}  In Mott scattering electrons with different spin interact with the atomic potential differently.  (b) \textit{Left:}  VLEED scattering off a target.  \textit{Right:}  In VLEED scattering a material with more unoccupied spin down states will preferentially absorb spin down electrons and reflect spin up electrons. \cite{Okuda-Kimura}}
  \label{fig:detectortypes}
\end{figure}

A newer method of measuring the spin polarization of electrons is very low energy electron diffraction (VLEED).
VLEED spin detectors use a ferromagnetic target to preferentially reflect spins of a chosen polarization.
This is accomplished because a magnetized ferromagnet has an asymmetry in the density of states with spin up and spin down.
The probability of an electron of a given spin to be reflected or absorbed by the target is proportional to the density of unoccupied spin states in the material \cite{Okuda-Kimura}; see Fig \ref{fig:detectortypes} (b).
This technique is much more efficient than Mott scattering, a typical figure of merit is $1\times10^{-2}$.
Furthermore, because paths A and B are identical, only one detector is required so detector asymmetry is not a problem in non magnetic samples.
Unfortunately, preparing the target is difficult and the scattering properties of the target degrade with time.
Great improvements were made by Bertacco \cite{Bertacco} who showed a Fe(001)-p(1x1)-O target was both a better polarizer and was orders of magnitude more stable than Fe alone.
Even with this improvement, VLEED instruments must still be calibrated against a Mott instrument and periodically checked to ensure stability.
Nonetheless, the two order of magnitude gain of VLEED detectors over Mott detectors is significant and has lead to much better SARPES results.
\section{COPHEE or ESPRESSO?}
While the number of SARPES experiments is growing, I will focus on the design of two of the more sophisticated instruments:  COPHEE (COmplete PHotoEmission Experiment) in the Swiss Light Source and ESPRESSO (Efficient SPin REsolved SpectroScopy Observation) located in Hiroshima Synchrotron Radiation Center.
Both of these resources combine spin resolved and spin integrated measurements.
This allows researchers to get a very detailed picture of the band structure with the spin integrated measurement before honing in on the spin texture.

\begin{figure}[h]
  \centering
  \includegraphics[height=2in]{fig/cophee.pdf}
  \caption[]
  {Schematic diagram of the COPHEE detector.  Photo-emitted electrons are energy and angle selected by a hemispherical analyzer.  After exiting the hemispherical analyzer the beam reaches a switch which sends the beam to one of two Mott analyzer.  The Mott analyzers are oriented such that all three spin projections can be measured \cite{Hoesch}.}
  \label{fig:cophee}
\end{figure}
COPHEE, shown in Fig. \ref{fig:cophee}, is unique because it employs two orthogonally mounted Mott polarimeters \cite{Hoesch}.
Both polarimeters are capable of measuring spin polarization in two directions.
As a result of this geometry it is able to measure four spin projections simultaneously, allowing for a full three dimensional spin characterization of the sample.
The redundant spin projection serves as a useful cross check between the two analysers.
COPHEE is is able to reach energy resolutions of 15meV and angular resolutions of $\pm1\degree$.

\begin{figure}[h]
  \centering
  \includegraphics[height=2in]{fig/espresso.pdf}
  \caption[]
  {Schematic diagram of the ESPRESSO detector.  The ESPRESSO detector uses a VLEED spin detector to measur the spin projection along the x or z axis \cite{Okuda}.}
  \label{fig:espresso}
\end{figure}
ESPRESSO, shown in Fig. \ref{fig:espresso}, abandons Mott polarimeters in favor of the more sensitive VLEED detector, as a result it has the highest resolution of any SARPES instrument to date \cite{Okuda}.
The VLEED polarimiter is a Fe(001)-p(1x1)-O film target and is operated at 6eV.
Electromagnetic coils are placed in both the x and z direction.
This allows the detector to measure both in plane and out of plane spin polarization of the sample being studied.
ESPRESSO is able to achieve energy resolutions of 7.5meV and angular resolutions of $\pm0.18\degree$.

\section{Experimental Results for Bismuth Films}
\begin{figure}[th]
  \centering
  \includegraphics[height=1.5in]{fig/bitop.pdf}
  \caption[]
  {The bismuth crystal structure from the top \cite{Shirasawa}.}
  \label{fig:bitop}
\end{figure}
\begin{figure}[th]
  \centering
  \includegraphics[height=1.5in]{fig/biside.pdf}
  \caption[]
  {The bismuth cyrstal structure from the side \cite{Shirasawa}.}
  \label{fig:biside}
\end{figure}
Bismuth (Bi) is the heaviest nonradioactive element.
Because it has such a large Z value it was expected to have strong spin-orbit coupling as well as strong spin-split surface states resulting from the Rashba effect.
Fig. \ref{fig:bitop} and Fig. \ref{fig:biside} show the crystal structure of Bi films that were used in these experiments.

The first SARPES study of Bi was conducted by Hirahara et al. \cite{Hirahara}.
They studied 10 bilayer thick films of Bi(111) using an unpolarized Helium discharge lamp (21.2eV) and a Mott detector to measure the spin asymmetry.
Their measurement had an energy resolution of 110meV and 1$\degree$ and probed surface states.
This measurement revealed the spin structure is antisymmetric with respect to the $\bar{\Gamma}$ point, see Fig. \ref{fig:hirahara}, as predicted by theory and the spin polarization could be as large as $\pm 0.5$.
\begin{figure}[h]
  \centering
  \includegraphics[height=2in]{fig/hirahara.pdf}
  \caption[]
  {The bismuth cyrstal structure from the side \cite{Hirahara}.}
  \label{fig:hirahara}
\end{figure}
Later, Kimura et al. used SARPES to probe the bulk continuum states of Bi(111) \cite{Kimura}.
Their findings are also consistent with \cite{Hirahara} and theoretical predictions of a strong Rashba splitting antisymmetric to the $\bar{\Gamma}$ point.

Using a Xenon discharge lamp (8.437eV), a group in Tohoku was able to achieve a resolution of 40meV on 40 bilayer samples \cite{Takayama}.
In addition to observing the same in plane Rashba splitting that previous groups reported, the Tohoku group observed an unexpected out-of-plane spin polarization.
The group also reported an asymmetry of the magnitude of the parallel component of polarization across the $\bar{\Gamma}$ point.
These features are shown in Fig. \ref{fig:tohoku}.
\begin{figure}[h]
  \centering
  \includegraphics[height=2in]{fig/tohoku.pdf}
  \caption[]
  {The stacking of Bi monolayers \cite{Takayama}.}
  \label{fig:tohoku}
\end{figure}
Both of these features are not described by the Rashba effect and were unexpected.
The authors offer the breaking of time reversal symmetry as a possible explanation for their observations.
Such a breaking would have profound implications because such symmetry breaking is usually caused by magnetic order in the crystal which is not present in Bi.

More recently the Hiroshima group studied Bi(111) using the ESPRESSO machine and a He lamp source (21.22eV) \cite{Miyahara}.
These measurements are the most accurate to date with resolutions of $\Delta E\sim20meV$ and $\Delta k\sim0.02\AA$.
The Hiroshima group found the expected in-plane Rashba helical spin structure as well as the anomalous out-of-plane spin polarization.
Unlike the Tohoku group, they were able to show a dependence of the out-of-plane spin polarization on the film quality.
Thus the Hiroshima group concluded the effect is a result of the three-fold crystal symmetry of Bi(111) films.
Their conclusions are supported by recent X-ray diffraction results that show Bi films can have two domain types in which the crystal axis is rotated 180 degrees to each other \cite{Shirasawa}.
The Hiroshima group was able to observe that the in-plane Rashba spin polarization was unaffected by domain type, while the out-of-plane spin polarization was opposite in the two domains.
\begin{figure}[th]
  \centering
  \includegraphics[height=2in]{fig/miyahara.pdf}
  \caption[]
  {The bismuth cyrstal structure from the side \cite{Miyahara}.}
  \label{fig:miyahara}
\end{figure}
\section{Outlook and Applications to Spintronics}
Recent improvements in SARPES have made it a powerful tool for studying spin-dependant phenomena in materials.
Consequently the number of SARPES experiments is growing and now commercial setups are sold.
Bismuth films are one of many recently studied systems where SARPES measurements have revealed features that were missed by other investigations.
Better understanding the role that the spin degrees of freedom play in solids will further our understanding of complicated many-body physics; giving us a better understanding of the world around us.
Additionally, better understanding spin degrees of freedom in solids can lead to novel electronics.
Perhaps most exciting applications of studying the surface spin properties of materials is the application of these materials for spintronics.
Spintronics are electronic devices that use spin dependant properties of materials and promise to increase speed, decrease power consumption, and shrink the size of current electronic devices \cite{wolf}.

\printbibliography
\end{document}
